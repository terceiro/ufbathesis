%% Template para dissertação/tese na classe UFBAthesis
%% versão 0.9.2
%% (c) 2005 Paulo G. S. Fonseca
%% (c) 2012 Antonio Terceiro
%% www.dcc.ufba.br/~terceiro/ufbathesis

\documentclass[phd, a4paper, classic]{ufbathesis}
\usepackage[utf8]{inputenc}

%% Preâmbulo:
%% coloque aqui o seu preâmbulo LaTeX, i.e., declaração de pacotes,
%% (re)definições de macros, medidas, etc.

\title{<TÍTULO DA OBRA>}
\date{<DATA DA DEFESA>}
\author{<NOME DO AUTOR>}
\adviser{<NOME DO(DA) ORIENTADOR(A)>}
\coadviser{<NOME DO(DA) CO-ORIENTADOR(A)>}

\begin{document}

% Folha de rosto
\dmccfrontpage{PMCC-Dsc-XXXX}
% Se seu trabalho não for uma tese de doutorado do DMCC, apague a linha
% acima e use \frontpage

%%
%% Parte pré-textual
%%
\frontmatter

% Portada (apresentação)
\dmccpresentationpage
% Se seu trabalho não for uma tese de doutorado do DMCC, apague a linha
% acima e use \presenationpage

% Ficha catalográfica
\authorcitationname{<SEU NOME EM CITAÇÕES>} % e.g. Terceiro, Antonio Soares de Azevedo
\advisercitationname{<NOME DO SEU ORIENTADOR EM CITAÇÕES>} % e.g. Chavez, Christina von Flach Garcia
\coadvisercitationname{<NOME DO SEU CO-ORIENTADOR EM CITAÇÕES>} % e.g. Mendonca, Manoel Gomes de
\catalogtype{<TIPO DE TRABALHO>} % e.g. ``Tese (doutorado)''
\catalogtopics{<TOPICOS PARA FICHA CATALOGRAFICA>} % e.g. ``1. Complexidade Estrutural. 2. Engenharia de Software''
\catalogcdd{<NUMERO CDD>} % e.g. ``CDD 20.ed. XXX.YY'' (esse número vai lhe ser dado pela biblioteca)
\catalogingsheet

% Termo de aprovação - exemplo
% Modifique com os membros da sua banca
\approvalsheet{Salvador, <DIA> de <MÊS> de <ANO>}{
  \comittemember{Profa. Dra. Professora 1}{Universidade XYZ}
  \comittemember{Prof. Dr. Professor 2}{Universidade 123}
  \comittemember{Profa. Dra. Professora 3}{Universidade ABC}
  \comittemember{Prof. Dr. Professor 4}{Universidade HJKL}
  \comittemember{Profa. Dra. Professora 5}{Universidade QWERTY}
}

% Agradecimentos
% Se preferir, crie um arquivo à parte e o inclua via \include{}
\acknowledgements
<DIGITE OS AGRADECIMENTOS AQUI>

% Resumo em Português
% Se preferir, crie um arquivo à parte e o inclua via \include{}
\resumo
<DIGITE O RESUMO AQUI>
% Palavras-chave do resumo em Português
\begin{keywords}
<DIGITE AS PALAVRAS-CHAVE AQUI>
\end{keywords}

% Resumo em Inglês
% Se preferir, crie um arquivo à parte e o inclua via \include{}
\abstract
% Palavras-chave do resumo em Inglês
\begin{keywords}
<DIGITE AS PALAVRAS-CHAVE AQUI>
\end{keywords}

% Sumário
% Comente para ocultar
\tableofcontents

% Lista de figuras
% Comente para ocultar
\listoffigures

% Lista de tabelas
% Comente para ocultar
\listoftables

%%
%% Parte textual
%%
\mainmatter

% É aconselhável criar cada capítulo em um arquivo à parte, digamos
% "capitulo1.tex", "capitulo2.tex", ... "capituloN.tex" e depois
% incluí-los com:
% \include{capitulo1}
% \include{capitulo2}
% ...
% \include{capituloN}
%
% Importante: Use \xchapter ao invés de \chapter, conforme exemplo abaixo.

\xchapter{Introdução}{Este é o primeiro capítulo, onde eu conto toda a
história deste trabalho}

Texto do capítulo \ldots

\xchapter{Revisão Bibliográfica}{Neste capítulo eu apresento todo o material que eu estudei durante a elaboração do trabalho}

Texto do capítulo \ldots

\backmatter

% Apêndices
% Comente se não houver apêndices
\appendix

% É aconselhável criar cada apêndice em um arquivo à parte, digamos
% "apendice1.tex", "apendice.tex", ... "apendiceM.tex" e depois
% incluí-los com:
% \include{apendice1}
% \include{apendice2}
% ...
% \include{apendiceM}


% Bibliografia
% É aconselhável utilizar o BibTeX a partir de um arquivo, digamos "biblio.bib".
% Para ajuda na criação do arquivo .bib e utilização do BibTeX, recorra ao
% BibTeXpress em www.cin.ufpe.br/~paguso/bibtexpress
\bibliographystyle{abnt-alf}
\bibliography{biblio}

%% Fim do documento
\end{document}
